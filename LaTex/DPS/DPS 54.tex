\PassOptionsToPackage{unicode}{hyperref}

\title{A Temporal Analysis of Cassini/VIMS Limb Brightness Profiles of Titan from 2004--2017}
\author{Aadvik Vashist \and Jason Barnes}
\date{October 20, 2022}

\documentclass[12pt]{article}
\usepackage{flushend}
\usepackage[margin=1in, top=0.5in, bottom=1in]{geometry}
\usepackage{natbib} % Add this line for citation management
\usepackage[hidelinks]{hyperref}

\begin{document}
\maketitle
\vspace{-1\baselineskip}

We document the evolution of the North-South Asymmetry (NSA) of Titan's haze albedo during the Cassini mission between 2004 and 2017. We use co-added visible and near-infrared images taken by the Cassini Visual and Infrared Mapping Spectrometer (VIMS) instrument from fourteen Titan flybys to enhance the contrast of the NSA. Over a half-Titan year, we observe a near-complete transition in the NSA boundary latitude across the geographic equator from the Southern to Northern hemisphere, including a 3-year loss in albedo contrast several years after the vernal equinox. The NSA disappearance matches observations of a reversal of the NSA in Hubble images before winter solstice between 1997-2000 \citep{lorenz2001titan} and in Space Telescope Imaging Spectrograph images between 2017-2019 \citep{karkoschka2022titan}. A comparison of NSA images taken at similar times but at different phase angles shows the NSA boundary is detectable, albeit with less contrast, at high phase angles (\raisebox{-0.5ex}{\textasciitilde}90 degs). We find several VIMS cube images further support a small yet detectable tilt between the super-rotating atmosphere and the solid body of Titan, as previously suggested in an analysis of 890 nm images from the Cassini Imaging Science Subsystem (ISS) by \cite{roman2009}.

\bibliographystyle{apalike} % This line sets the bibliography style
\bibliography{NSAreferences.bib} % This line includes your bibliography file

\end{document}
